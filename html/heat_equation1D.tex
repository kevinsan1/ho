
% This LaTeX was auto-generated from an M-file by MATLAB.
% To make changes, update the M-file and republish this document.

\documentclass{article}
\usepackage{graphicx}
\usepackage{color}

\sloppy
\definecolor{lightgray}{gray}{0.5}
\setlength{\parindent}{0pt}

\begin{document}

    
    
\subsection*{Contents}

\begin{itemize}
\setlength{\itemsep}{-1ex}
   \item Initialize Source function
   \item * Set up the Laplacian operator matrix
   \item * Initialize Q-matrix
   \item * Compute A-matrix (Tn+1)=ATn
   \item * Initialize loop and plot variables
   \item * Loop over desired number of steps
   \item Plot
\end{itemize}
\begin{verbatim}
% heat_equation - Program to solve the diffusion equation
% using the Backward Euler method
clear; help heat_equation; % Clear memory and print header

%* Initialize parameters (time step, grid spacing, etc.)
tau = 1e-4; % Enter time step
N = 100; % Number of grid points
L = 1; % The system extends from (x)=(0) to (x)=(L)
h = L/N;
i = 0:(N-1);
x = h/2 + i*h;
w = 0.2;
xs = 0.5;
ys = 0.5;
\end{verbatim}

        \color{lightgray} \begin{verbatim}  heat_equation - Program to solve the diffusion equation
  using the Backward Euler method

\end{verbatim} \color{black}
    

\subsection*{Initialize Source function}

\begin{verbatim}
S = zeros(N); % Set all elements to zero
xExponent = (x'-xs).^2;
S = exp(-xExponent/w^2);
deltaFunction = zeros(N,1);
deltaFunction(round(N/2))=2;
\end{verbatim}


\subsection*{* Set up the Laplacian operator matrix}

\begin{verbatim}
lap = zeros(N);  % Set all elements to zero
coeff = 1/h^2;
for i=2:(N-1)
    lap(i,i-1) = coeff;
    lap(i,i) = -2*coeff;  % Set interior rows
    lap(i,i+1) = coeff;
end
% Boundary conditions
lap(1,1)=-coeff;
lap(1,2)=coeff;
lap(N,N)=-coeff;
lap(N,N-1)=coeff;
\end{verbatim}


\subsection*{* Initialize Q-matrix}

\begin{verbatim}
Q = deltaFunction;
\end{verbatim}


\subsection*{* Compute A-matrix (Tn+1)=ATn}

\begin{verbatim}
dM = eye(N) - tau*lap;
\end{verbatim}


\subsection*{* Initialize loop and plot variables}

\begin{verbatim}
max_iter = .5/tau;
time = linspace(0,max_iter*tau,max_iter);      % Record time for plots
Qplot(:,1) = Q; % initial value
\end{verbatim}


\subsection*{* Loop over desired number of steps}

\begin{verbatim}
for iter=2:round(.25/tau)
    %* Compute new temperature
    Q = dM\(Q)+deltaFunction;
    Qplot(:,iter) = Q(:);
end
for iter=round(.25/tau):max_iter
    %* Compute new temperature
    Q = dM\(Q);
    Qplot(:,iter) = Q(:);
end
\end{verbatim}


\subsection*{Plot}

\begin{par}
figure(2);clf; mesh(time,x,Qplot); xlabel('t (s)'); ylabel('x (m)'); \%\% Print Plots saveFigurePath = '/Users/kevin/SkyDrive/KTH Work/LaTeX Reports/Heat Equation/Figures/'; \%\% Plot 1 set(figure(2), 'PaperPositionMode', 'auto'); print('-depsc2', [saveFigurePath ...     sprintf('deltaFunctionPlot')]);
\end{par} \vspace{1em}



\end{document}
    
